% !TEX encoding = UTF-8 Unicode
% !TEX TS-program = XeLaTeX
\documentclass{article}

\usepackage{fontspec}   %加這個就可以設定字體
\usepackage{xeCJK}       %讓中英文字體分開設置
\setCJKmainfont{BiauKai} %設定中文為系統上的字型,而英文不去更動,使用原TeX字型
\setmainfont{Times} % 設定英文字型
\XeTeXlinebreaklocale "zh"             %這兩行一定要加,中文才能自動換行
\XeTeXlinebreakskip = 0pt plus 1pt     %這兩行一定要加,中文才能自動換行

\usepackage{titling}
\setlength{\droptitle}{-12em} % 將標題移動至頁面的上面


\title{試試看玩玩看!}
\author{郭昌易}
\date{} %不要日期

\begin{document}
\maketitle

\section{題目}

接下來將提供五道題目給予讀者進行練習,並在稍候附上提供的小提示以及參考解答。

\begin{table}[h]
\caption{題目清單}

\begin{center}
\begin{tabular}{||l||}
\hline
A.將車子右駕的行駛方向改為左邊駕駛\\
\hline
B. 車子可以單獨沿著一條膠帶行駛\\
\hline
C. 改變手把控制車子的方式\\
\hline
D. 自行設計一個FSM流程並實現\\
\hline
E. 車子能夠辨識一個物件並做出相對應動作\\
\hline
\end{tabular}
\end{center}
\end{table}

\subsection{題目A}

原本車子的行進方向為右邊駕駛,行進時白色膠帶在車子的右邊、黃色膠帶在車子的左邊。此題則希望將右邊駕駛改為左邊駕駛,意即車子行進時,黃色膠帶在車子的右邊,而白色膠帶在車子的左邊。

\begin{itemize}

\item 建議可以往與處理顏色有關的節點為考慮的出發點,例如line-detector或者lane-filter。

\end{itemize}

\subsection{題目B}

在一開始的設定裡,車子行進時需要搭配黃線與白線才能行駛,意即車子移動時會同時受到白線與黃線的影響。此題希望將車子轉變成可以沿著單條任意顏色的膠帶行走,並且不受其他顏色的膠帶干擾。

\begin{itemize}

\item 建議可以往與處理顏色有關的節點為考慮的出發點,例如line-detector或者lane-filter。

\end{itemize}

\subsection{題目C}

此題希望能夠修改搖桿上的按鍵配置,以方便讀者更加熟悉ROS的實作。例如原先使用類比搖桿控制車子的前後左右,現在改為透過十字鍵來操控。

\begin{itemize}

\item 建議可以查閱joy-mapper的節點為參考出發點。
\end{itemize}

\subsection{題目D}

此題希望能夠使讀者自行設計一個完整的FSM流程。例如車子遇到紅線會停下來並且左轉(由行進中的狀態轉為左轉狀態),或是使用遙桿自由切換自動駕駛以及手動駕駛兩種狀態。

\begin{itemize}

\item 建議可以參考FSM章節的介紹。

\end{itemize}

\subsection{題目E}

此題希望能夠使得車子俱備偵測物件的能力,並且與偵測到的物件進行互動。例如車子在自動駕駛時,若是遇上了一隻鴨子,則車子需停下來並且閃避,當確認前方無障礙物後,再進行自動駕駛。

\begin{itemize}

\item 建議可以先行研讀object-detector的節點以及FSM的實作方法後,再進行。

\end{itemize}

\end{document}